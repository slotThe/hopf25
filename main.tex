\documentclass[aspectratio=169,12pt,professionalfont]{beamer}
\usepackage[type=presentation,font=palatino,osf,math=fancy]{styles/style}
\usetheme{m}
\bibliography{bib}

\renewcommand*{\hom}[1]{\ensuremath{\lfloor#1\rfloor}}
\newcommand*{\cohom}[1]{\ensuremath{\lceil#1\rceil}}
\newcommand*{\Ind}{\mathsf{Ind}}
\newcommand*{\bbcomod}[1][B]{\Tetramod[#1]}
\renewcommand*{\kvect}{\mathsf{vect}}
\renewcommand*{\kVect}{\mathsf{Vect}}

\newcommand*{\pageBase}{tony-zorman.com/hopf25}
\newcommand*{\website}{https://\pageBase}

\title{Reconstruction for Lax Module Monads}
\msubtitle{Foregoing the fibre functor in \(n\) easy steps!}
\msubtitlelink{Based on joint work with Matti Stroiński:
  \href{https://arxiv.org/abs/2409.00793}{\texttt{arXiv:2409.00793}}%
}
\author[Tony Zorman]{\centering{%
    Tony Zorman \\
    \footnotesize\texttt{\href{mailto:tony.zorman@tu-dresden.de}{tony.zorman@tu-dresden.de}}}}
\date{25.04.2025}
\institute{%
  Technische Universität Dresden%
  \hfil\textbullet\hfil%
  Zellescher Weg 12--14%
  \hfil\textbullet\hfil%
  01069 Dresden%
  \hfil\textbullet\hfil%
  Germany%
  \hfil\hfil%
}
\qrCode{\website}

\begin{document}

\maketitle

\begin{frame}[standout]
  Given a monoidal category \(\cat{C}\),
  are all left \(\cat{C}\)-module categories
  equivalent to the modules of
  an algebra object in \(\cat{C}\)?
\end{frame}

\begin{frame}
  \begin{theorem}[{\cite{ostrik03:modul-hopf,Etingof2015}}]\label{finiterecon}
    Let \(\cat{C}\) be a \alert<2>{finite} \alert<3,4>{tensor category}
    and let \(\cat{M}\) be a \alert<2>{finite} abelian \(\cat{C}\)\hyp{}module category,
    such that the evaluation functor \(\blank \triangleright \ell \from \cat{C} \to \cat{M}\) is \alert<3>{exact},
    for all \(\ell \in \cat{M}\).
    Then there exists an algebra object \(A \in \cat{C}\)
    such that there is an equivalence of \(\cat{C}\)\hyp{}module categories \(\mathrm{mod}_{\cat{C}}(A) \simeq \cat{M}\).
  \end{theorem}
  \pause%
  Finiteness assumptions,
  \pause%
  exactness assumptions,
  \pause%
  and \alert<5>{rigidity assumptions}.
  \pause\pause%
  \begin{proposition*}[{\cite{douglas19}}]
    In the absence of rigidity,
    there are finite abelian \(\cat{C}\)-module categories that cannot be realised as the modules of an algebra object in \(\cat{C}\).
  \end{proposition*}
\end{frame}

\begin{frame}[standout]
  Given a monoidal category \(\cat{C}\),
  are all left \(\cat{C}\)-module categories
  equivalent to the modules of
  a \alert{monad} on \(\cat{C}\)?\vphantom{algebraobject}
\end{frame}

\begin{frame}\frametitle{The setup}
  Let \(\cat{C}\) be a \textcolor{gray}{\(\Bbbk\)-linear} monoidal,
  and \(\cat{M}, \cat{N}\) \textcolor{gray}{\(\Bbbk\)-linear} left \(\cat{C}\)-module categories.
  \pause%
  \begin{align*}
    \otimes \from \cat{C} \kotimes \cat{C} \to \cat{C},
    \qquad\qquad
    \lact \from \cat{C} \kotimes \cat{M} \to \cat{M}.
  \end{align*}
  \pause%
  such that for all \(x, y, z \in \cat{C}\) and \(m \in \cat{M}\), \eg,
  \[
    (x \otimes y) \otimes z \cong x \otimes (y \otimes z)
    \qquad\text{and}\qquad
    (x \otimes y) \lact \ell \cong x \lact (y \lact \ell).
  \]

  \pause%
  A functor \(F \from \cat{M} \to \cat{N}\) is a \emph{lax \(\cat{C}\)-module functor}
  if there exists \textcolor{gray}{an appropriately associative and unital} natural transformation
  \[
    F_2\from \blank \lact F(\bblank) \nt F(\blank \lact \bblank).
  \]
  \pause%
  The functor \(F\) is \emph{oplax} if \(F_2\) goes the other way,
  and \emph{strong} if it is invertible.
\end{frame}

\begin{frame}\frametitle{The Yoneda lemma™}
  \begin{proposition*}
    There is an equivalence of \(\cat{C}\)-module categories
    \begin{equation*}
      \cat{M} \simeq \mathsf{Str}\cat{C}\mathsf{Mod}(\cat{C},\cat{M}),\qquad
      \pause
      \ell \mapsto \alert<5>{\blank \lact \ell},\qquad
      \pause
      F1 \mapsfrom F.
    \end{equation*}
    \pause%
    In particular, \(\cat{C}^{\textcolor{gray}{\tensorop}} \simeq \mathsf{Str}\cat{C}\mathsf{Mod}(\cat{C},\cat{C})\).
  \end{proposition*}
  \pause%
\end{frame}

\begin{frame}[standout]
  Study cases in which \(\blank \lact \ell\) admits a right adjoint.
  \pause%
  The resulting monad canonically has a lax \(\cat{C}\)-module structure.
  \pause%
  Then apply Beck's monadicity theorem.
\end{frame}

\begin{frame}\frametitle{Kelly and Beck}
  \begin{theorem}[Kelly's doctrinal adjunctions,~\alt<-2>{\cite{kelly74:doctr}}{\cite{kelly74:doctr,halbig23:diagr-comod-monad}}]
    Given an adjunction
    \alt<-2>{\(\adj{F}{U}{\cat{C}}{\cat{D}}\)}{\(\adj{F}{U}{\cat{M}}{\cat{N}}\)}
    between
    \alt<-2>{\alert<2>{monoidal} categories}{\(\cat{C}\)-\alert<3>{module} categories},
    oplax
    \alt<-2>{\alert<2>{monoidal}}{\(\cat{C}\)-\alert<3>{module}}
    structures on \(F\) are in bijective correspondence with
    lax \alt<-2>{\alert<2>{monoidal}}{\(\cat{C}\)-\alert<3>{module}}
    structures on \(U\).
  \end{theorem}
  \pause\pause\pause%
  An adjunction \(\adj{F}{U}{\cat{C}}{\cat{M}}\) is called \emph{monadic} if the canonical comparison functor
  \(K\from \cat{M} \to \cat{C}^{UF}\) is an equivalence.
  \pause%
  \begin{theorem}[\alt<6>{Abelian monadicity,~\cite{ben-zvi18:integ}}{Beck's monadicity theorem}]
    An adjunction is monadic if and only if \(U\)\!
    \alt<6>{is exact and reflects zero objects}{creates \(U\)-split coequalisers}.
  \end{theorem}
\end{frame}

\begin{frame}\frametitle{Internal projectives}
  Let \(\cat{C}\) and \(\cat{M}\) be abelian.
  \pause%
  An object \(\ell \in \cat{M}\) is \emph{closed} if there is an adjunction
  \[
    \adj{\blank \lact \ell}{\hom{\ell, \blank}}{\cat{C}}{\cat{M}}.
  \]
  \pause%
  A closed object is called \emph{\(\cat{C}\)-projective} if \(\hom{\ell, \blank}\) is (right) exact
  \pause%
  and a \emph{\(\cat{C}\)-generator} if it is faithful.
  \pause%
  \begin{example}
    \begin{itemize}
      \item Every object in a rigid monoidal category \(\cat{C}\) is \(\cat{C}\)-projective.\pause%
      \item Finite \(\cat{C}\)-module categories over finite tensor categories always\\
      admit \(\cat{C}\)-projective \(\cat{C}\)-generators~\cite{Etingof2015,douglas19}.
    \end{itemize}
  \end{example}
\end{frame}

\begin{frame}[standout]
  Only the Eilenberg–Moore category of an oplax \(\cat{C}\)-module monad has a canonical \(\cat{C}\)-module structure.
\end{frame}

\begin{frame}\frametitle{Linton coequalisers}
  \begin{definition}
    The \emph{Linton coequaliser} of \(x \in \cat{C}\) and \(m \in \cat{M}^T\) is:
    \[
      \begin{tikzcd}[ampersand replacement=\&]
        {T(x \lact Tm)} \& {T(x \lact m)} \& {x \blact m.}
        \arrow["\raisebox{0.3em}{\(T(x \,\lact\, \nabla_{m})\)}", shift left=1, from=1-1, to=1-2]
        \arrow["\raisebox{-0.3em}{\(\mu_{x \lact m}\circ TT_{\mathsf{a};x,m}\)}"', shift right=1, from=1-1, to=1-2]
        \arrow[two heads, from=1-2, to=1-3]
      \end{tikzcd}%
    \]
  \end{definition}
  \pause%
  \begin{theorem}[{\cite{stroinski2024:reconstr}}]
    The Eilenberg–Moore category of any right exact lax \(\cat{C}\)-module monad
    can be equipped with a canonical \(\cat{C}\)-module structure by mean of Linton coequalisers.
  \end{theorem}
\end{frame}

\begin{frame}\frametitle{The reconstruction result}
  \begin{theorem}[{\cite{stroinski2024:reconstr}}]
    Let \(\cat{C}\) be an abelian monoidal category,
    \(\cat{M}\) an abelian \(\cat{C}\)\hyp{}module category,
    and assume that \(\ell \in \cat{M}\) is a closed \(\cat{C}\)\hyp{}projective \(\cat{C}\)\hyp{}generator.
    \pause%
    Then there is an equivalence of \(\cat{C}\)\hyp{}module categories
    \vspace{-0.2cm}\[
      \cat{M} \simeq \cat{C}^{\hom{\ell, \blank \,\lact\, \ell}}.
    \]\vspace{-0.2cm}
    \pause%
    Furthermore, there is a bijection
    \vspace{-0.2cm}\begin{align*}
      \faktor{\{(\cat{M},\ell)\ \text{as before}\}}{\cat{M} \simeq \cat{N}}
      &\xleftrightarrow{\ \cong\ }
        \Bigg\{
        \begin{aligned}
          &\text{Right exact lax \(\cat{C}\)\hyp{}module}\\
          &\text{monads on \(\cat{C}\)}
        \end{aligned}
        \Bigg\}
        /\, \cat{C}^T \!\simeq\! \cat{C}^S \\
      (\cat{M},\ell) &\longmapsto \hom{\ell,-\lact \ell} \\
      (\cat{C}^T, T1) &\longmapsfrom T
    \end{align*}
  \end{theorem}
\end{frame}

\begin{frame}\frametitle{Back to Hopf algebras}
  Let \(H\) be a Hopf algebra and \(\cat{C} \defeq \tetramodfd[H]{}\)
  \pause%
  \(\implies \Ind(\cat{C}) \simeq \Tetramod[H]{}\).
  \pause%

  Let \(\cat{M}\) be an abelian \(\cat{C}\)\hyp{}module category
  such that \(\Ind(\cat{M})\)
  admits a coclosed \(\Ind(\cat{C})\)\hyp{}injective \(\Ind(\cat{C})\)\hyp{}cogenerator.
  \pause%

  Then there exists an \(H\)\hyp{}comodule coalgebra \(C\)
  such that \(\Ind(\cat{M}) \simeq \mathsf{Comod}_{H}C\) as \(\Ind(\cat{C})\)\hyp{}module categories.
  \pause%

  This restricts to a \(\cat{C}\)\hyp{}module equivalence \(\cat{M} \simeq \mathsf{comod}_{H}C\).
\end{frame}

\section{(Op)lax module functors in action}

\begin{frame}\frametitle{Hopf trimodules}
  \begin{theorem}[{\cite{stroinski2024:reconstr}}]
    Let \(B\) be a bialgebra, and define \(\cat{V} \defeq \bComod\).
    \pause%
    There is a monoidal equivalence
    \begin{align*}
      \Trimod &\to \mathsf{LexfLax}\cat{V}\mathsf{Mod}(\cat{V}, \cat{V})\\
      X &\mapsto (X \cotens_B \blank, \chi)
    \end{align*}
    between the category of Hopf trimodules, and
    the category of left exact finitary lax \(\cat{V}\)\hyp{}module endofunctors on \(\cat{V}\).
  \end{theorem}
\end{frame}

\begin{frame}\frametitle{The interchanger}
  For all \(M, N \in \bbcomod\),
  the arrow
  \[
    \chi_{M, N} \from M \kotimes (X \cotens_B N) \to X \cotens_B (M \kotimes N)
  \]
  \pause%
  is defined by
  \[
    \tikzfig{def:bicomodule-interchange}
  \]
\end{frame}

\begin{frame}\frametitle{Deducing a theorem for Hopf trimodules}
  \begin{proposition*}
    Let \(\cat{C}\) be a left closed monoidal category such that every lax \(\cat{C}\)\hyp{}module endofunctor of \(\cat{C}\) is strong%
    —in other words, that the monoidal embedding
    \[
      \mathsf{Str}\cat{C}\mathsf{Mod}(\cat{C},\cat{C}) \longhookrightarrow \mathsf{Lax}\cat{C}\mathsf{Mod}(\cat{C},\cat{C})
    \]
    is an equivalence.
    Then \(\cat{C}\) is left rigid.
  \end{proposition*}
  \pause%
  \begin{corollary}[{\cite{stroinski2024:reconstr}}]
    A bialgebra \(B\) admits a twisted antipode if and only if the canonical functor
    \(B \kotimes \blank\from \Tetramod[B] \to \Tetramod[B][B][][B]\) is an equivalence.
  \end{corollary}
\end{frame}

\begin{frame}\frametitle{Fusion operators for Hopf monads}
  \begin{proposition*}[{\cite{stroinski2024:reconstr}}]
    Let \(\adj{F}{U}{\cat{C}}{\cat{D}}\) be an oplax monoidal adjunction.
    \pause%
    The strong monoidal structure of \(U\) turns \(\cat{C}\) into a \(\cat{D}\)\hyp{}module category via \(\blank \lact \bblank \;\defeq\; U(\blank) \otimes \bblank\).
    \pause%
    The bimonad \(T \defeq UF\) on \(\cat{C}\) becomes an oplax \(\cat{D}\)\hyp{}module monad.
    \pause%
    In particular, the right fusion operator is the ``free part'' of the coherence morphism:
    \[
      T_{2;F,\Id} = T_{\mathsf{rf}}.
    \]
    \pause%
    Further, \(T_{\mathsf{rf}}\) is an isomorphism if and only if \(T_2\) is.
  \end{proposition*}
\end{frame}

\begin{frame}\frametitle{Thanks!}
  \centering
  \begin{minipage}[bt]{0.4\textwidth}
    \qrcode[hyperlink,height=3.5cm]{\website}
  \end{minipage}\begin{minipage}[bt]{0.5\textwidth}
    \raggedright%
    \href{\website}{\texttt{\pageBase}}\\[1cm]
    \href{https://arxiv.org/abs/2409.00793}{%
      Reconstruction of module categories
      in the infinite and non-rigid settings.
      \texttt{arXiv:2409.00793}}
  \end{minipage}
\end{frame}

\begin{frame}[allowframebreaks]\frametitle{References}
  \nocite{hausser99:integ-theor-quasi-hopf-algeb}
  \nocite{Moerdijk2002}
  \nocite{deligne02:categ}
  \nocite{ostrik03:modul-hopf}
  \nocite{ostrik04:tensor-p}
  \nocite{Bruguieres2011}
  \nocite{saracco17:hopf}
  \nocite{halbig23:diagr-comod-monad}
  \printbibliography[heading=none]%
\end{frame}

\end{document}

%%% Local Variables:
%%% mode: latex
%%% TeX-master: t
%%% End:
